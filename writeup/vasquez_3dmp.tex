\documentclass[]{article}

%opening
\title{3D motion planning problem}
\author{Irving Vasquez}

\begin{document}

\maketitle

\begin{abstract}

\end{abstract}

\section{Explain the starter code}

motion\_planning.py is the main script. It is a child class from Drone class. It is in charge of communicating with the drone and it reacts depending on the events coming from the drone. 

The class follows the event programming paradigm. When the program starts, in \texttt{state\_callback()} function the drone goes from MANUAL state to ARMING, PLANNING, TAKE\_OFF, WAYPOINT, LANDING and DISARMING.

For planning the path the \texttt{plan\_path()} function is used. Inside this function an arbitrary goal is defined. Then, A* is used to plan the path to such goal. A* uses an euclidean distance and all the actions have the same cost and they move the drone only in a 2D plane.

\section{Found Issues}

I found that when many waypoints are send to be displayed the program broke dawn.


\end{document}
